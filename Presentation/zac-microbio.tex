\documentclass[12pt,a4paper,oneside]{article}
\usepackage[utf8]{inputenc}
\usepackage[french]{babel}
\usepackage[T1]{fontenc}
\usepackage{graphicx}
\usepackage{amsmath}
\usepackage[colorlinks]{hyperref}
\author{Thomas~Liautaud\\Pharmacien des hôpitaux\\Grand Hôpital de l'Est Francilien}
\date{2 juillet 2018}
\title{Un serveur web couplé à un système de gestion de base de données pour l'enregistrement des données de la surveillance microbiologique d'une zone d'atmosphère contrôlée}

\begin{document}
\maketitle
\thanks{\begin{center}
projet pour le \\DU Bionformatique et Datamanagement\\ Ioannis Nicolis\\Université Paris-Descartes
\end{center}}
\tableofcontents
\section{Introduction}
Les mélanges de nutrition parentérale sont utilisés pour les patients dont les apports en nutriments par voie orale sont impossibles ou insuffisants. Ces mélanges sont des préparations parentérales pour perfusion au sens de la Pharmacopée européenne\footnote{Pharmacopée européenne 9\up{ème} edition, EDQM}: elles sont stériles et apyrogènes.

Pour la préparation des médicaments stériles les Bonnes pratiques de préparation \footnote{Bonnes Pratiques de Préparations, Afssaps, 2007} (BPP) imposent des  exigences  particulières en vue de réduire les risques de contamination microbienne, particulaire et pyrogène. Elles définissent notamment les zone d'atmosphère contrôlée (ZAC) qui sont des locaux et équipements dont les qualités microbiologique et particulaire sont contrôlées (figure \ref{zac}).

\begin{figure}[h]
\caption{\label{zac}Schéma des principes d'une zone d'atmosphère contrôlée pour la préparation de médicaments stériles}
\includegraphics[scale=0.4]{zac.png}
\end{figure}

Aux fins de préparation de médicaments stériles, 4 classes de ZAC sont distinguées. Des recommandations pour la surveillance microbiologique des ZAC sont présentées dans la table \ref{zac}. Les méthodes d'échantillonnages utilisent des boîtes de Petri, des échantillons volumétriques d'air et des contrôles de surface (prélevés au moyen de géloses de contact et/ou écouvillons).
\begin{table}[h]
\caption{Recommandations pour la surveillance microbiologique des zones d'atmosphère controlée en activité. Limites de contamination en unité formant colonie (ufc).\label{zac}}
\begin{center}
\begin{tabular}{|c|c|c|c|c|}
	\hline
	\textbf{Classe} & \textbf{Air} &\textbf{Surfaces} & \textbf{Gants}\\
	\hline
	A & <1 & <1 & <1\\
	B & 10 & 5 & 5\\
	C & 50 & 25 & -\\
	D & 100 & 50 & -\\
	
	\hline
\end{tabular}
\end{center}
\end{table}

La pharmacie de notre établissement prépare des mélanges de nutrition parentérale pour la réanimation néonatale. L'unité de préparation des mélanges de nutrition parentérale de notre établissement est constituée d'un isolateur en classe A dans une zone de préparation en classe D. Afin de surveiller la conformité de notre unité aux exigences des BPP, des prélèvements microbiologiques sont effectués quotidiennement. Une quarantaine de points de prélèvements ont été définis :
\begin{itemize}
 \item Selon le lieu : isolateur ou salle;
 \item Selon le type de prélèvement : air, surface, gant;
 \item Selon le dispositif utilisé : boîte de Petri, gélose contact, écouvillon;
 \item Selon le point précis de prélèvement : gants, champs stériles, balance, automate de fabrication, etc.
 \end{itemize}
Moins d'une dizaine de points sont prélevés quotidiennement. Afin de prélever tous les points chaque semaine, un planning de prélèvement répartit les points à prélever sur l'ensemble de la semaine.

Les prélèvements microbiologiques sont traités par le laboratoire d'hygiène du service de Biologie. En cas de contamination microbiologique de l'isolateur, les résultats (en nombre de colonies) sont communiqués immédiatement par téléphone au pharmacien. L'identification des germes est faite ultérieurement. Dès l'identification des germes, un compte rendu signé par un biologiste est rendu à la pharmacie.

Actuellement les prélèvements effectués et leurs résultats sont enregistrés sur un tableur.
\section{Objectif}
L'objectif de ce projet est de mettre à disposition des utilisateurs un serveur web couplé à système de gestion de base de données qui doit permettre de saisir et consulter les prélèvements microbiologiques et leurs résultats.
\section{Materiel et méthode}

\subsection{Système de gestion de base de données}
 \subsubsection{Base de données postgreSQL}

La base de données postgreSQL (version 9.3) est installée sur le système d'exploitation GNU/Linux (Linux Mint 17.3).
La base de données est nommée \emph{bacterio\_upnp}.


\subsubsection{Création des tables}
\begin{description}
\item[Table disp\_prelev] (Table \ref{disp}) référence les dispositifs de prélèvement. Elle a été créée avec la requête :
\begin{tabbing}
CREATE TABLE disp\_prelev(\\
id SERIAL PRIMARY KEY,\\
disp VARCHAR(50) NOT NULL\\
);
\end{tabbing}
\begin{table}
\caption{Table disp\_prelev \label{disp}}
\begin{center}
\begin{tabular}{|c|c|c|c|}
	\hline
	\textbf{Nom de variable} & \textbf{Type} & \textbf{Clé primaire} & \textbf{Clé étrangère}\\
	\hline
	id & SERIAL & oui &\\
	disp & VARCHAR(50)& &\\
	\hline
\end{tabular}
\end{center}
\end{table}

\item[Table limites\_classes] (Table \ref{classes}) liste les classes microbiologiques, les types de prélèvements et les limites microbiologiques. Elle a été créée avec la requête :
\begin{tabbing}
CREATE TABLE limites\_classes(\\
id SERIAL PRIMARY KEY,\\
classe VARCHAR(10) NOT NULL\\
type VARCHAR(50) NOT NULL\\
limite INTEGER NOT NULL\\
);
\end{tabbing}
\begin{table}
\caption{Table limites\_classes \label{classes}}
\begin{center}
\begin{tabular}{|c|c|c|c|}
	\hline
	\textbf{Nom de variable} & \textbf{Type} & \textbf{Clé primaire} & \textbf{Clé étrangère}\\
	\hline
	id & SERIAL & oui &\\
	classe & VARCHAR(10)& &\\
	type & VARCHAR(50)& &\\
	limite & INTEGER & &\\
	\hline
\end{tabular}
\end{center}
\end{table}

\item[Table points\_prelev] (Table \ref{points}) liste les points de prélèvements utilisés. Elle a été créée avec la requête :
\begin{tabbing}
CREATE TABLE points\_prelev(\\
id SERIAL PRIMARY KEY,\\
point VARCHAR(10) NOT NULL,\\
id\_disp INTEGER NOT NULL REFERENCES disp\_prelev(id)\\
id\_class INTEGER NOT NULL REFERENCES limites\_classes(id),\\
);
\end{tabbing}
\begin{table}
\caption{Table points\_prelev \label{points}}
\begin{center}
\begin{tabular}{|c|c|c|c|}
	\hline
	\textbf{Nom de variable} & \textbf{Type} & \textbf{Clé primaire} & \textbf{Clé étrangère}\\
	\hline
	id & SERIAL & oui &\\
	point & VARCHAR(10)& &\\
	id\_disp & INTEGER & & disp\_prelev(id)\\
	id\_class & INTEGER & & limites\_classes(id)\\
	\hline
\end{tabular}
\end{center}
\end{table}

\item[Table jours\_prelev] (Table \ref{jours}) liste les jours de prélèvements. Elle a été créée avec la requête :
\begin{tabbing}
CREATE TABLE jours\_prelev(\\
id SERIAL PRIMARY KEY,\\
jour VARCHAR(10) NOT NULL,\\
);
\end{tabbing}
\begin{table}
\caption{Table jours\_prelev \label{jours}}
\begin{center}
\begin{tabular}{|c|c|c|c|}
	\hline
	\textbf{Nom de variable} & \textbf{Type} & \textbf{Clé primaire} & \textbf{Clé étrangère}\\
	\hline
	id & SERIAL & oui &\\
	jours & VARCHAR(10)& &\\
	\hline
\end{tabular}
\end{center}
\end{table}


\item[Table planning\_prelev] (Table \ref{planning}) associe des jours de prélèvements et des points de prélèvements. Elle a été créée avec la requête :
\begin{tabbing}
CREATE TABLE planning\_prelev(\\
id\_jour INTEGER NOT NULL REFERENCES jours\_prelev(id),\\
id\_point INTEGER NOT NULL REFERENCES points\_prelev(id),\\
PRIMARY KEY (id\_jour, id\_point)
);
\end{tabbing}
\begin{table}
\caption{Table planning\_prelev \label{planning}}
\begin{center}
\begin{tabular}{|c|c|c|c|}
	\hline
	\textbf{Nom de variable} & \textbf{Type} & \textbf{Clé primaire} & \textbf{Clé étrangère}\\
	\hline
	id\_jour & INTEGER & oui & jours\_prelev(id)\\
	id\_point & INTEGER & oui & points\_prelev(id)\\
	\hline
\end{tabular}
\end{center}
\end{table}

\item[Table prelevements] (Table \ref{prelevements}) liste les prélèvements réalisés. Elle a été créée avec la requête :
\begin{tabbing}
CREATE TABLE prelevements(\\
id SERIAL PRIMARY KEY,\\
date\_prelev DATE NOT NULL,\\
id\_point INTEGER NOT NULL REFERENCES points\_prelev(id)
);
\end{tabbing}
\begin{table}
\caption{Table prelevements \label{prelevements}}
\begin{center}
\begin{tabular}{|c|c|c|c|}
	\hline
	\textbf{Nom de variable} & \textbf{Type} & \textbf{Clé primaire} & \textbf{Clé étrangère}\\
	\hline
	id & SERIAL & oui & \\
	date\_prelev & DATE & &\\
	id\_point & INTEGER & & points\_prelev(id)\\
	\hline
\end{tabular}
\end{center}
\end{table}

\item[Table resultats] (Table \ref{resultats}) pour enregistrer les résultats des prélèvements. Elle a été créée avec la requête :

\begin{tabbing}
CREATE TABLE resultats(\\
id SERIAL PRIMARY KEY,\\
id\_prelev INTEGER NOT NULL REFERENCES prelevements(id),\\
date\_res DATE NOT NULL\\
\nopagebreak
ufc INTEGER NOT NULL,\\
germe VARCHAR(100),\\
type\_rendu VARCHAR(10) NOT NULL
);
\end{tabbing}
\begin{table}
\caption{Table resultats \label{resultats}}
\begin{center}
\begin{tabular}{|c|c|c|c|}
	\hline
	\textbf{Nom de variable} & \textbf{Type} & \textbf{Clé primaire} & \textbf{Clé étrangère}\\
	\hline
	id & SERIAL & oui & \\
	id\_prelev & INTEGER & & prelevements(id) \\
	date\_res & DATE & & \\
	ufc & INTEGER & & \\
	germe & VARCHAR(100) & & \\
	type\_rendu & VARCHAR(10) & & \\
	\hline
\end{tabular}
\end{center}
\end{table}

\end{description}

\subsubsection{Peuplement des tables}
Les enregistrements des tables ont été renseignés par des commandes SQL \emph{INSERT}, sauf pour les tables \emph{prelevements} et \emph{resultats}. Ces deux tables ont été renseignées par l'importation de fichiers .csv contenant les données déjà enregistrées sur tableur et par des enregistrements de prélèvements et résultats fictifs à partir des pages web.

Les prélèvements et résultats présentés sont fictifs et ne représentent en aucun cas des résultats et prélèvements réels.
\subsection{Serveur Web}
Le serveur web Apache (version 2.4.7) et installé sur le système d'exploitation GNU/Linux (Linux Mint 17.3).
Les pages Web sont écrites en langage HTML et PHP (version 7.2.4-1)
\section{Résultats}
\subsection{Base de données}
Le dump de la base de données est fourni sous le nom de fichier : \textbf{bacterio\_upnp.sql}. Toutes les requêtes sont faites avec l'utilisateur \emph{pharmacien} et le mot de passe \emph{zac}.

Le schéma relationnel de la base \emph{bacterio\_upnp} (figure \ref{erd}) a été obtenu par DBVizualizer.
\begin{figure}
\caption{\label{erd}Schéma relationnel de bacterio\_upnp}
\includegraphics[scale=0.4]{erd.png}
\end{figure}

\subsection{Pages web}

Les fichiers .php et .html du site web (figure \ref{arbo}) sont les suivants :
\begin{figure}[h]
\caption{\label{arbo}Arborescence des pages web}
\includegraphics[scale=0.45]{arborescence.png}
\end{figure}
\begin{description}
\item[accueil.html :] Page centrale avec les liens vers les autres pages du site : accès aux différentes pages du site web par des liens hypertexte ; présentation du site web et de la base de données.
\item[planning.php :] Extraction du planning de prélèvement depuis la base de données bacterio\_upnp. Présentation sous forme de tableau.
\item[planning\_select.html :] Définition des filtres du planning de prélèvement avec la méthode POST.
\item[planning\_select.php :] Planning des points de prélèvements sélectionnés avec planning\_select.html.
\item[prelevements.php :] Tableau des prélèvements et des résultats extraits de la bdd bacterio\_upnp.
\item[prelevement\_select.html :] Sélection des filtres des prélèvements et résultats de la bdd bacterio\_upnp par la méthode POST.
\item[prelevements\_select.php :] Affichage des tables prelevements et resultats de la bdd bacterio\_upnp avec les filtres sélectionnés par la page prelevements\_select.html.
\item[palmares\_prelevements.php :] Affichage du classement du nombre de prélèvements par point.
\item[palmares\_prelevements\_select.html] : Filtres de sélection pour les prélèvements par la méthode POST.
\item[palmares\_prelevements\_select.php :] Affichage du tableau du nombre de prélèvements par point filtré par palmares\_prelevements\_select.html
\item[palmares\_resultats.php :] Présentation d'un tableau du nombre de résultats par point.
\item[resultats\_select.html :] Filtres pour le palmares des résultats par la méthode POST.
\item[palmares\_resultats\_select.php :] Présentation d'un tableau du nombre de résultats par point filtré par resultats\_select.html.
\item[palmares\_germes.php :] Classement par ordre décroissants du nombre de résults par germe.
\item[germes\_select.html :] Sélection des filtres pour le palmares des germes par la méthode POST.
\item[palmares\_germes\_select.php :] Utilisation des filtres de germes\_select.html pour afficher le palmares du nombre de résulats par germe.
\item[prelevements\_saisie.html] Filtres pour la saisie des prélèvements par la méthode POST.
\item[prelevements\_saisie.php :] Affichage des points à prélever selon les filtres sélectionnés dans prelevements\_saisie.html.
\item[prelevements\_saisie\_confirm.php :] Enregistrement des prélèvements sélectionnés dans prelevements\_saisie.php dans la table prelevements.
\item[resultats\_saisie.php :] Saisie des résultats sélectionnés avec resultats\_saisie.html.
\item[resultats\_saisie\_confirm.php :] Enregistrements des résultats saisis dans la page resultats\_saisie.php.
\item[zac-microbio.pdf :] Présentation du projet et description de la base de données et du site web (ce document).
\item[tables\_brutes.php :] Présentation sous forme de tableaux des tables de la base de données bacterio\_upnp.


\end{description}

\section{Discussion et conclusion}
Ce site web couplé à une base de données permet de consulter les données enregistrées. Il permet également d'enregistrer de nouveaux prélèvements et leurs résultats microbiologiques.

Il manque la possibilité de modifier les prélèvements et les résultats enregistrés en cas d'erreur de saisie. Il serait également interessant de filtrer les prélèvements qui n'ont pas encore de résultats microbiologiques.

Ce site web couplé à une base de données pourra être testé en conditions réelles lorsque un serveur web sera mis à disposition de la pharmacie de l'hôpital. Le retour des utilisateurs le fera encore évolué.

\end{document}
