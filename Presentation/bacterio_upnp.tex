\documentclass[12pt,a4paper,oneside]{article}
\usepackage[utf8]{inputenc}
\usepackage[french]{babel}
\usepackage[T1]{fontenc}
\usepackage{graphicx}
\author{Thomas~Liautaud\\Pharmacien des hôpitaux\\Grand Hôpital de l'Est Francilien}
\date{26 avril 2018}
\title{Surveillance microbiologique d'une unité de préparation de nutrition parentérale : gestion des enregistrements avec un serveur web et un système de gestion de base de données}

\begin{document}
\maketitle
\thanks{\begin{center}
DU Bionformatique et Datamanagement\\ Ioannis Nicolis\\Université Paris-Descartes
\end{center}}
\tableofcontents
\section{Introduction}
Les mélanges de nutrition parentérale sont utilisés pour les patients dont les apports en nutriments par voie orale sont impossibles ou insuffisants. Ces mélanges sont des préparations parentérales pour perfusion au sens de la Pharmacopée européenne\footnote{Pharmacopée européenne 9\up{ème} edition, EDQM}: elles sont stériles et apyrogènes.

Pour la préparation des médicaments stériles les Bonnes Pratiques de préparation \footnote{Bonnes Pratiques de Préparations, ANSM, 2007} (BPP) imposent des  exigences  particulières en vue de réduire les risques de contamination microbienne, particulaire et pyrogène. Elles définissent notamment les zones à atmosphère contrôlée (ZAC) qui sont des locaux et équipements dont les qualités microbiologique et particulaire sont contrôlées.

Aux fins de préparation de médicaments stériles, 4 classes de ZAC sont distinguées. Des recommandations pour la surveillance microbiologique des ZAC sont présentées dans la table \ref{classes}. Les méthodes d'échantillonnages utilisent des boîtes de Petri, des échantillons volumétriques d'air et des contrôles de surface (prélevés au moyen de géloses de contact et/ou écouvillons).
\begin{table}[h]
\caption{Recommandations pour la surveillance microbiologique des zones d'atmosphère controlée en activité. Limites de contamination en unité formant colonie (ufc) .\label{classes}}
\begin{center}
\begin{tabular}{|c|c|c|c|c|}
	\hline
	\textbf{Classe} & \textbf{Air} &\textbf{Surfaces} & \textbf{Gants}\\
	\hline
	A & <1 & <1 & <1\\
	B & 10 & 5 & 5\\
	C & 50 & 25 & -\\
	D & 100 & 50 & -\\
	
	\hline
\end{tabular}
\end{center}
\end{table}

La pharmacie de notre établissement prépare des mélanges de nutrition parentérale pour la néonatologie. L'unité de préparation des mélanges de nutrition parentérale de notre établissement est constituée d'un isolateur en classe A dans une zone de préparation en classe D. Afin de surveiller la conformité de notre unité aux exigences des BPP, des prélèvements microbiologiques sont effectués quotidiennement. Une quarantaine de points de prélèvements ont été définis :
\begin{itemize}
 \item Selon le lieu : isolateur ou salle;
 \item Selon le type de prélèvement : air, surface, gant;
 \item Selon le dispositif utilisé : boîte de Petri, gélose contact, écouvillon;
 \item Selon le point précis de prélèvement : tel gant, tel champ stérile, tel point de la balance...
 \end{itemize}
Moins d'une dizaine de points sont prélevés quotidiennement. Afin de prélever tous les points chaque semaine, un planning de prélèvement répartit les points à prélever sur l'ensemble de la semaine.

Les prélèvements microbiologiques sont traités par le laboratoire d'hygiène du service de Biologie. En cas de contamination microbiologique de l'isolateur, les résultats (en nombre de colonies) sont communiqués immédiatement par téléphone au pharmacien. L'identification des germes est faite ultérieurement. Dès l'identification des germes, un compte rendu signé par un biologiste est rendu à la pharmacie.

Actuellement les prélèvements effectués et leurs résultats sont enregistrés sur un tableur.
\section{Objectif}
L'objectif de ce projet est d'enregistrer sur une base de données relationnelle les prélèvements biologiques et leurs résultats. Les données seront mises à disposition des utilisateurs sur un serveur Web. Ce serveur web doit permettre aux utilisateurs de saisir et consulter les prélèvements et les résultats.
\section{Materiel et méthode}

\subsection{Base de données}
La base de données postgreSQL (version 9.3) est installée sur le système d'exploitation GNU/Linux (Linux Mint 17.3).

\subsection{Serveur Web}
Le serveur web Apache (version 2.4.7) et installé sur le système d'exploitation GNU/Linux (Linux Mint 17.3).
Les pages Web sont écrites en langage HTML et PHP (version 7.2.4-1)

\section{Résultats}
\subsection{Base de données postgreSQL}
La base de données est nommée \emph{bacterio\_upnp}. Le dump de la base de données est fourni sous le nom de fichier : \textbf{bacterio\_upnp.sql}.


\subsubsection{Description des tables}

La table disp\_prelev (Table \ref{disp}) référence les dispositifs de prélèvement. Elle a été créée avec la requête :

\texttt{CREATE TABLE disp\_prelev(id SERIAL PRIMARY KEY, disp VARCHAR(50) NOT NULL);}

\begin{table}[h]
\caption{Table disp\_prelev \label{disp}}
\begin{center}
\begin{tabular}{|c|c|c|c|}
	\hline
	\textbf{Nom de variable} & \textbf{Type} & \textbf{Clé primaire} & \textbf{Clé étrangère}\\
	\hline
	id & SERIAL & oui &\\
	disp & VARCHAR(50)& &\\
	\hline
\end{tabular}
\end{center}
\end{table}

\subsubsection{Relations entre les tables}
\subsection{Serveur Web}
Les fichiers .php et .html du site web sont les suivants :
\begin{description}
\item[accueil.html :] Présentation de la base de données et accès aux différentes pages par des liens hypertexte.

\end{description}


\section{Discussion}
Discussion
\section{Conclusion}
conclusion





\end{document}
